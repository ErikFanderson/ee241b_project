%%%%%%%%%%%%%%%%%%%%%%%%%%%%%%%%%%%%%%%%%
% Journal Article
% LaTeX Template
% Version 1.4 (15/5/16)
%
% This template has been downloaded from:
% http://www.LaTeXTemplates.com
%
% Original author:
% Frits Wenneker (http://www.howtotex.com) with extensive modifications by
% Vel (vel@LaTeXTemplates.com)
%
% License:
% CC BY-NC-SA 3.0 (http://creativecommons.org/licenses/by-nc-sa/3.0/)
%
%%%%%%%%%%%%%%%%%%%%%%%%%%%%%%%%%%%%%%%%%

%----------------------------------------------------------------------------------------
%	PACKAGES AND OTHER DOCUMENT CONFIGURATIONS
%----------------------------------------------------------------------------------------

\documentclass[twoside,twocolumn]{article}

\usepackage{blindtext} % Package to generate dummy text throughout this template 

\usepackage[sc]{mathpazo} % Use the Palatino font
\usepackage[T1]{fontenc} % Use 8-bit encoding that has 256 glyphs
\linespread{1.05} % Line spacing - Palatino needs more space between lines
\usepackage{microtype} % Slightly tweak font spacing for aesthetics

\usepackage[english]{babel} % Language hyphenation and typographical rules

\usepackage[hmarginratio=1:1,top=32mm,columnsep=20pt]{geometry} % Document margins
\usepackage[hang, small,labelfont=bf,up,textfont=it,up]{caption} % Custom captions under/above floats in tables or figures
\usepackage{booktabs} % Horizontal rules in tables

\usepackage{lettrine} % The lettrine is the first enlarged letter at the beginning of the text

\usepackage{enumitem} % Customized lists
\setlist[itemize]{noitemsep} % Make itemize lists more compact

\usepackage{abstract} % Allows abstract customization
\renewcommand{\abstractnamefont}{\normalfont\bfseries} % Set the "Abstract" text to bold
\renewcommand{\abstracttextfont}{\normalfont\small\itshape} % Set the abstract itself to small italic text

\usepackage{titlesec} % Allows customization of titles
\renewcommand\thesection{\Roman{section}} % Roman numerals for the sections
\renewcommand\thesubsection{\roman{subsection}} % roman numerals for subsections
\titleformat{\section}[block]{\large\scshape\centering}{\thesection.}{1em}{} % Change the look of the section titles
\titleformat{\subsection}[block]{\large}{\thesubsection.}{1em}{} % Change the look of the section titles

\usepackage{fancyhdr} % Headers and footers
\pagestyle{fancy} % All pages have headers and footers
\fancyhead{} % Blank out the default header
\fancyfoot{} % Blank out the default footer
\fancyhead[C]{UC Berkeley Advanced Digital Integrated Circuits Letters $\bullet$ April 2020} % Custom header text
\fancyfoot[RO,LE]{\thepage} % Custom footer text

\usepackage{titling} % Customizing the title section

\usepackage{hyperref} % For hyperlinks in the PDF

%----------------------------------------------------------------------------------------
%	TITLE SECTION
%----------------------------------------------------------------------------------------

\setlength{\droptitle}{-4\baselineskip} % Move the title up

\pretitle{\begin{center}\Huge\bfseries} % Article title formatting
\posttitle{\end{center}} % Article title closing formatting
\title{Implementation of NEMS Interconnect Relays for Reconfigurable Circuits} % Article title
\author{%
\textsc{Lars Tatum}\\[1ex] % Your name
\normalsize University of California \\ % Your institution
\normalsize \href{mailto:lpt@berkeley.edu}{lpt@berkeley.edu} % Your email address
\and % Uncomment if 2 authors are required, duplicate these 4 lines if more
\textsc{Erik Anderson}\\[1ex] % Second author's name
\normalsize University of California \\ % Your institution
\normalsize \href{mailto:efa@eecs.berkeley.edu}{efa@eecs.berkeley.edu} % Second author's email address
}
\date{\today} % Leave empty to omit a date
\renewcommand{\maketitlehookd}{%
\begin{abstract}
\noindent In recent years, there has been much attention on bringing mechanical 
systems to the nanoscale. Reconfigurable computing for AI and Machine Learning has 
also sparked lots of growth in the tech industry. Recently, CMOS compatible 
Back-End-Of-Line mechanical systems have been proposed to create denser reconfigurable
systems. In this project, we propose the design of a BEOL 5T standard cell switch and 
its implementation in a complex digital system for reduced die area and increased circuit
complexity in a 7nm process (ASAP7). Example applications where these would be most pertinent
include reconfigurable digital logic, reconfigurable hi-fidelity analog frontend, or 
compute-in-memory for AI/ML. Our goal is to perform the first step of Design-Technology
Co-Optimization by integrating the switch standard cell into the hammer-vlsi flow alongside 
other devices from the ASAP7 library.
\end{abstract}
}

%----------------------------------------------------------------------------------------

\begin{document}

% Print the title
\maketitle

%----------------------------------------------------------------------------------------
%	ARTICLE CONTENTS
%----------------------------------------------------------------------------------------

\section{Introduction}
Field Programmable Gate Arrays (FPGAs) have long been used to enable rapid prototyping of digital systems pre-tapeout of Application Specific Integrated Circuits (ASICs). The framework they provide for high-performance reconfigurable digital systems is now proving useful for end-use applications including ML-acceleration, cryptocurrency mining, high-frequency stock trading, and software-defined radios. FPGA demand is predicted to continue rising as these protoyping and end-use markets increase in demand.

To enable this flexibility, FPGA-based systems are much more demanding than their ASIC counterparts. FPGAs come with larger die area requirements, delay, and dynamic power consumption. These factors increase the cost of FPGA-based systems and lower their potential performance over ASIC-based systems. Recently, CMOS compatible devices have been developed that could mitigate these area/power/performance drawbacks.  Nano-Electro-Mechanical-Systems (NEMS) provide a promising, easily-integratable way to further FPGA-based systems. But first, a look at where all this overhead comes from in FPGAs.

\textit{The Island-Style FPGA architecture}

FPGAs general consist of a general logic fabric as well as application-specific components, such as block RAM, DSPs, clocking, I/O, and serial I/O. This project will tackle the general logic fabric, since it makes up the majority of the FPGA.

Island-Style logic fabric can be divided into two major components: Configurable Logic Blocks and Reconfigurable Interconnects. Configurable Logic Blocks (CLBs) are used to implement arbitrary combinational and sequential logic circuits. They consist of Lookup Tables, Flip Flops, and Multiplexers. Each CLB can be programmed to perform an arithmetic or state-based portion of a larger logic system. CLBs are connected together using reconfigurable interconnects. Reconfigurable Interconnects, made up of Connection Blocks and Switch Boxes, are the “breadboard” of the FPGA. They allow the portions of the digital system to communicate. In an Island-Style Architecture, Connection Blocks connect CLBs to routing wires that run throughout the fabric. Those wires are then connected to each other via Switch Boxes. These routing elements are primarily made up of muxes and SRAM, a setup that is readily implementable using NEM relays.

These Reconfigurable Interconnects dominate the area/power overhead incurred by FPGAs [Altera 08]. In an ASIC, interconnects are simple, passive elements; circuits are connected using fixed metal lines paths above the transistor devices. To enable design flexibility, FPGAs add SRAM configuration storage as well as the mux circuitry required to connect blocks and wires. Not only does this additional circuitry incur a large power/delay overhead, it also must be implemented completely in the Front-End-Of-Line alongside logic transistors, adding significantly to die area.

Obviously, FPGAs are awesome, and have many compelling applications. But how to mitigate the performance/area/power hits? ASIC circuits are connected solely via Back-End-Of-Line (BEOL) Metal lines. If FPGAs could emulate this structure, much of the overhead incurred from reconfigurable interconnects could be eliminated.


\textit{NEM Relays}

Nanoelectromechanical Relays are electrically actuated mechanical switches. Notably, they experience zero leakage, have the potential for lower on-state resistance than NMOS pass gate, and can be implemented in the BEOL in a CMOS-compatible process flow. The major disadvantages of NEM relays are large mechanical switching delay (> 1ns) and lower reliability with respect to transistors (~10^9-10^10 cycles). The two relay device architectures examined in this work are lateral “crab style” relays and vertical interconnect relays. Due to their hysteretic properties, they can be implemented to replace both the interconnect pass-gates and their corresponding SRAM cells.

The lateral “crab-style” relays are the result of many years of development at Stanford and UC Berkeley to create an efficient CMOS compatible NEM relay. These relays are switched on/off using the gate to body voltage difference. When “off”, no current flows across the air gap. When “on”, current flows between the drain and source with low on-state resistance. With this design, the device can be body-biased to have an extremely small switching voltage, experimentally shown to be sub- 50mV range.

The vertical relays are designed to be more compact and compatible with existing CMOS process flows. Rather than using deposited metal or polysilicon as the structural material, these vertical relays use the already-present metal vias/interconnects in a CMOS process flow. These relays can be implemented as single-pole double-throw switches or simpler single-pole single-throw switches (like the lateral relays). Both have their own drawbacks and benefits. While the lateral relays involve a “custom” structure, these vertical ones are designed within the constraints of standard CMOS BEOL design rules; i.e, they use standard metal line/via structures, and dimensions are standard for the process node.

%------------------------------------------------

\section{Proposed Techniques}\label{sec:proposed}

%%%IN PROGRESS%%%


%------------------------------------------------

\section{Methods (A/B comparison)}
SRAM-based island-style FPGAs represent the vast majority of commercial and academic
FPGA implementations \cite{farooq_fpga_2012}. For this reason, this architecture 
will be used to construct three functionally equivalent FPGAs using the three different
routing schemes discussed in the previous Section \ref{sec:proposed}. 

% Describe tile 
To create an island-style FPGA a basic \textit{tile} must first be constructed. 
Each tile consists of one \textit{configurable logic block} (CLB), 
two \textit{connection boxes} (CBs), and one \textit{switch box} (SB). 
The CLB contains N basic logic elements (BLEs) which each contain a K-input 
look-up-table (LUT) and a flip-flop. In addition, CLBs provide an intra-cluster 
routing network to connect the CLB I/O to the individual BLE I/O. The CBs connect
the CLBs to their adjacent routing channels, while the SBs provide the connections
between intersecting routing channels. It is common to fold the output portion of 
the CBs into the SB \cite{chen_efficient_2010}. Thus, the same MUXes used for
connecting routing channels can be enlarged to accomodate the additional 
CLB output connections.

% Describe specific FPGA architecture that we will use 
To ensure that our results are consistent with previous works, we
will create a very similar tile to the one described in \cite{chen_efficient_2010}. 
This tile consists of 10 4-LUT BLEs w/ 22 input pins and 10 output pins.
In order to simplify the design, the CBs in our tile will connect both
the outputs and inputs of the CLB to the adjacent routing channel.
The intra-CLB routing network will be fully populated, allowing for connection 
between any BLE input and CLB input as well as between any BLE output and CLB output.
The SB flexibility will be set to 3 ($f_{SB} = 3$), allowing all 
incoming signals to the SB to be switched left, right, or straight through. 
The CB input and output flexibilities will be set to 0.2 ($f_{CB-in} = 0.2$) 
and 0.1 respectively ($f_{CB-out} = 0.1$). These flexibilities determine the 
proportion of signals within the adjacent routing channels that the CB can 
connect an input or output signal to.

% Detail CMOS implementation (mainly mux architecture) 
The CMOS-only FPGA will serve as a baseline for the NEMs designs.
Because the programmable interconnect (PI) will be replaced with NEMs relays, the make up
of our baseline's PI will significantly impact the perceived benefits. 
We will use simple NMOS pass-transistors to build the MUXes used 
for the PI and the LUTs. While transmission gates show promise for scaled nodes, 
commercial FPGAs continue to use pass-transistors with gate-boosting and internal buffering
to achieve a slightly lower area-delay product \cite{chiasson_should_2013}.

% Detail how we will model the NEMs relays
The NEMs relays can be modeled as simple RC circuits when configured in the 
ON-state \cite{chen_efficient_2010} \cite{chen_integrated_2008} and can be ignored, 
because they are physically disconnected from the 
circuit, when in the OFF-state. Modeling the dynamics of the NEMs relays as they switch on and
off is not as important as this steady-state behavior for this analysis. 
As long as the configuration times are not prohibitively long, and they have been 
shown not to be \cite{??} \cite{??}, then the dynamics of the switches can be ignored.

% Detail final analysis that will be done
Preliminary area analysis can be done by comparing each FPGA tile.
It is important to account for the additional metal area used by the 
relays as this will significantly impact the place-and-route quality
due to higher routing congestion. Further analysis can be performed 
on the full FPGAs to determine the dissipated power and maximum 
operating frequency for each design for a given set of test circuits.

%------------------------------------------------

%------------------------------------------------
\section{Hypothesis}

%------------------------------------------------

\section{Conclusion}

%----------------------------------------------------------------------------------------
%	REFERENCE LIST
%----------------------------------------------------------------------------------------

\bibliographystyle{plain}
\bibliography{midterm_report}

%----------------------------------------------------------------------------------------

\end{document}
