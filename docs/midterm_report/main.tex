%%%%%%%%%%%%%%%%%%%%%%%%%%%%%%%%%%%%%%%%%
% Journal Article
% LaTeX Template
% Version 1.4 (15/5/16)
%
% This template has been downloaded from:
% http://www.LaTeXTemplates.com
%
% Original author:
% Frits Wenneker (http://www.howtotex.com) with extensive modifications by
% Vel (vel@LaTeXTemplates.com)
%
% License:
% CC BY-NC-SA 3.0 (http://creativecommons.org/licenses/by-nc-sa/3.0/)
%
%%%%%%%%%%%%%%%%%%%%%%%%%%%%%%%%%%%%%%%%%

%----------------------------------------------------------------------------------------
%	PACKAGES AND OTHER DOCUMENT CONFIGURATIONS
%----------------------------------------------------------------------------------------

\documentclass[twoside,twocolumn]{article}

\usepackage{blindtext} % Package to generate dummy text throughout this template 

\usepackage[sc]{mathpazo} % Use the Palatino font
\usepackage[T1]{fontenc} % Use 8-bit encoding that has 256 glyphs
\linespread{1.05} % Line spacing - Palatino needs more space between lines
\usepackage{microtype} % Slightly tweak font spacing for aesthetics

\usepackage[english]{babel} % Language hyphenation and typographical rules

\usepackage[hmarginratio=1:1,top=32mm,columnsep=20pt]{geometry} % Document margins
\usepackage[hang, small,labelfont=bf,up,textfont=it,up]{caption} % Custom captions under/above floats in tables or figures
\usepackage{booktabs} % Horizontal rules in tables

\usepackage{lettrine} % The lettrine is the first enlarged letter at the beginning of the text

\usepackage{enumitem} % Customized lists
\setlist[itemize]{noitemsep} % Make itemize lists more compact

\usepackage{abstract} % Allows abstract customization
\renewcommand{\abstractnamefont}{\normalfont\bfseries} % Set the "Abstract" text to bold
\renewcommand{\abstracttextfont}{\normalfont\small\itshape} % Set the abstract itself to small italic text

\usepackage{titlesec} % Allows customization of titles
\renewcommand\thesection{\Roman{section}} % Roman numerals for the sections
\renewcommand\thesubsection{\roman{subsection}} % roman numerals for subsections
\titleformat{\section}[block]{\large\scshape\centering}{\thesection.}{1em}{} % Change the look of the section titles
\titleformat{\subsection}[block]{\large}{\thesubsection.}{1em}{} % Change the look of the section titles

\usepackage{fancyhdr} % Headers and footers
\pagestyle{fancy} % All pages have headers and footers
\fancyhead{} % Blank out the default header
\fancyfoot{} % Blank out the default footer
\fancyhead[C]{Running title $\bullet$ May 2016 $\bullet$ Vol. XXI, No. 1} % Custom header text
\fancyfoot[RO,LE]{\thepage} % Custom footer text

\usepackage{titling} % Customizing the title section

\usepackage{hyperref} % For hyperlinks in the PDF

%----------------------------------------------------------------------------------------
%	TITLE SECTION
%----------------------------------------------------------------------------------------

\setlength{\droptitle}{-4\baselineskip} % Move the title up

\pretitle{\begin{center}\Huge\bfseries} % Article title formatting
\posttitle{\end{center}} % Article title closing formatting
\title{Implementation of NEMS Interconnect Relays for Reconfigurable Circuits} % Article title
\author{%
\textsc{Lars Tatum}\\[1ex] % Your name
\normalsize University of California \\ % Your institution
\normalsize \href{mailto:lpt@berkeley.edu}{lpt@berkeley.edu} % Your email address
\and % Uncomment if 2 authors are required, duplicate these 4 lines if more
\textsc{Erik Anderson}\\[1ex] % Second author's name
\normalsize University of California \\ % Your institution
\normalsize \href{mailto:efa@eecs.berkeley.edu}{efa@eecs.berkeley.edu} % Second author's email address
}
\date{\today} % Leave empty to omit a date
\renewcommand{\maketitlehookd}{%
\begin{abstract}
\noindent TODO: roughly consolidate with bora 5-sentence guide. 1) Subject area 
2)Contrast 3)Key contribution 4)Basis for comparison 5)Summary of (expected) results
In recent years, there has been much attention on bringing mechanical 
systems to the nanoscale. Reconfigurable computing for AI and Machine Learning has 
also sparked lots of growth in the tech industry. Recently, CMOS compatible 
Back-End-Of-Line mechanical systems have been proposed to create denser reconfigurable
systems. In this project, we propose the design of a BEOL 5T standard cell switch and 
its implementation in a complex digital system for reduced die area and increased circuit
complexity in a 7nm process (ASAP7). Example applications where these would be most pertinent
include reconfigurable digital logic, reconfigurable hi-fidelity analog frontend, or 
compute-in-memory for AI/ML. Our goal is to perform the first step of Design-Technology
Co-Optimization by integrating the switch standard cell into the hammer-vlsi flow alongside 
other devices from the ASAP7 library.
\end{abstract}
}

%----------------------------------------------------------------------------------------

\begin{document}

% Print the title
\maketitle

%----------------------------------------------------------------------------------------
%	ARTICLE CONTENTS
%----------------------------------------------------------------------------------------

\section{Introduction}

%------------------------------------------------

\section{Proposed Techniques}\label{sec:proposed}

%------------------------------------------------

\section{Methods (A/B comparison)}
SRAM-based island-style FPGAs represent the vast majority of commercial and academic
FPGA implementations \cite{farooq_fpga_2012}. For this reason, this architecture 
will be used to construct three functionally equivalent FPGAs using the three different
routing schemes discussed in the previous Section \ref{sec:proposed}. 

% Describe tile 
To create an island-style FPGA a basic \textit{tile} must first be constructed. 
Each tile consists of one \textit{configurable logic block} (CLB), 
two \textit{connection boxes} (CBs), and one \textit{switch box} (SB). 
The CLB contains N basic logic elements (BLEs) which each contain a K-input 
look-up-table (LUT) and a flip-flop. In addition, CLBs provide an intra-cluster 
routing network to connect the CLB I/O to the individual BLE I/O. The CBs connect
the CLBs to their adjacent routing channels, while the SBs provide the connections
between intersecting routing channels. It is common to fold the output portion of 
the CBs into the SB \cite{chen_efficient_2010}. Thus, the same MUXes used for
connecting routing channels can be enlarged to accomodate the additional 
CLB output connections.

% Describe specific FPGA architecture that we will use 
To ensure that our results are consistent with previous works, we
will create a very similar tile to the one described in \cite{chen_efficient_2010}. 
This tile consists of 10 4-LUT BLEs w/ 22 input pins and 10 output pins.
In order to simplify the design, the CBs in our tile will connect both
the outputs and inputs of the CLB to the adjacent routing channel.
The intra-CLB routing network will be fully populated, allowing for connection 
between any BLE input and CLB input as well as between any BLE output and CLB output.
The SB flexibility will be set to 3 ($f_{SB} = 3$), allowing all 
incoming signals to the SB to be switched left, right, or straight through. 
The CB input and output flexibilities will be set to 0.2 ($f_{CB-in} = 0.2$) 
and 0.1 respectively ($f_{CB-out} = 0.1$). These flexibilities determine the 
proportion of signals within the adjacent routing channels that the CB can 
connect an input or output signal to.

% Detail CMOS implementation (mainly mux architecture) 
The CMOS-only FPGA will serve as a baseline for the NEMs designs.
Because the programmable interconnect (PI) will be replaced with NEMs relays, the make up
of our baseline's PI will significantly impact the perceived benefits. 
We will use simple NMOS pass-transistors to build the MUXes used 
for the PI and the LUTs. While transmission gates show promise for scaled nodes, 
commercial FPGAs continue to use pass-transistors with gate-boosting and internal buffering
to achieve a slightly lower area-delay product \cite{chiasson_should_2013}.

% Detail how we will model the NEMs relays
The NEMs relays can be modeled as simple RC circuits when configured in the 
ON-state \cite{chen_efficient_2010} \cite{chen_integrated_2008} and can be ignored, 
because they are physically disconnected from the 
circuit, when in the OFF-state. Modeling the dynamics of the NEMs relays as they switch on and
off is not as important as this steady-state behavior for this analysis. 
As long as the configuration times are not prohibitively long, and they have been 
shown not to be \cite{??} \cite{??}, then the dynamics of the switches can be ignored.

% Detail final analysis that will be done
Preliminary area analysis can be done by comparing each FPGA tile.
It is important to account for the additional metal area used by the 
relays as this will significantly impact the place-and-route quality
due to higher routing congestion. Further analysis can be performed 
on the full FPGAs to determine the dissipated power and maximum 
operating frequency for each design for a given set of test circuits.

%------------------------------------------------

%------------------------------------------------
\section{Hypothesis}

%------------------------------------------------

\section{Conclusion}

%----------------------------------------------------------------------------------------
%	REFERENCE LIST
%----------------------------------------------------------------------------------------

\bibliographystyle{plain}
\bibliography{midterm_report}

%----------------------------------------------------------------------------------------

\end{document}
